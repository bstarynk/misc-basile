\documentclass[a4paper,12pt]{article} %no more twocolumn
\usepackage{times}
\usepackage{hevea}
\usepackage[hyphens]{url}
\usepackage{hyperref}
\usepackage{lastpage}
\usepackage{fancyhdr}
\pagestyle{fancy}
\title{QISH introduction \\ }


\author{Basile STARYNKEVITCH \\
\mailto{basile@starynkevitch.net} \\
\url{http://www.starynkevitch.net/Basile/} \\
\small{8, rue de la Fa�encerie, 92340 Bourg La Reine, France}
}

\newcommand{\qishallocate}{\tt \verb+qish_allocate+}

\begin{document}
\def\UrlFont{\tt\footnotesize}

\maketitle

{\small \tt
\begin{verbatim}
$Id: qishintro.tex 19 2005-05-08 17:35:50Z basile $
\end{verbatim}
}
\newpage
\tableofcontents
\newpage

Please be nice to send me an email if you use this information and
this Qish software.

Qish is available from 
% $Format:"\\mbox{\\url{http://www.starynkevitch.net/Basile/qish-$ReleaseVersion$.tar.gz}}"$
\mbox{\url{http://www.starynkevitch.net/Basile/qish-1.0pre1.tar.gz}} 
%
(as a gnuzipped source tarball) 
and the latest snapshot is on \mbox{\url{http://www.starynkevitch.net/Basile/qish-snapshot.tar.gz}} 
and this document is on
\mbox{\url{http://www.starynkevitch.net/Basile/qishintro.html}}. See also my
home page on \url{http://www.starynkevitch.net/Basile/} or Qish page
\url{http://freshmeat.net/projects/qish/} on Freshmeat for
announcement of newer versions.

Documentation should be rewritten. Multithreading is not fully working
yet in version 0.9. Stay tuned.

A mailing list (not yet archived) is available as
{\mailto{qish@lists.apinc.org}} send an email to
{\mailto{qish-subscribe@lists.apinc.org}} or to me at
{\mailto{basile@starynkevitch.net}} for subscription.

Qish is developed on a PC/x86 running Linux. It could be portable to
other Unixes machines\footnote{On some machines you'll need to flush
the register stack in {\tt qish\_garbagecollect} - for example,
Sparc machines might need {\tt ta 3} to flush registers}. Qish could
be ported to x86 under Windows, but I don't want to do that.

To compile Qish on a strictly conforming ISO C 1999 compiler use the
{\tt -DSTRICT\_C99} compile flag\footnote{With this flag, Qish compiles
  without warnings both on GNU {\tt gcc-3.2} with {\tt -pedantic
    -std=c99} and on {\tt tcc} - see \url{http://tinycc.org/}}.  But
Qish requires that successful pointer arguments are located
consecutively (upwards or downwards) in memory (which is not
guaranteed or even meaningful in general for ISO C 1999 compilers.).

The machine dependent parts of Qish are carefully coded, but not
tested elsewhere. If you have access to other machine architecture,
please tell me if you succeeded in compiling, porting, and running
this software. Qish requires that successive pointer arguments and
pointer fields are consecutive in memory.

Boehm's conservative garbage collector on
\url{http://www.hpl.hp.com/personal/Hans_Boehm/gc/} should be easier
to use\footnote{Boehm's GC has an interface compatible with {\tt
malloc}, is compatible with threads and finalization, and is used in
GCC-3 for the Java runtime}, but does not compact memory, as any
copying GC (like Qish) does. See the section \ref{sec:bench} for a
small (and not very significant) benchmark running both Boehm's and
Qish GC (and explicit malloc and free).


This GC is not mostly copying\footnote{Unlike Bartlett's GC -1990- US
patent 4,907,151 - thanks to Gerd Moellmann for the reference}.  It
copies every movable object (even if it is on the call stack - in that
case the variable or argument should be known to our GC thru the {\tt
BEGIN\_LOCAL\_FRAME} or equivalent macro), and mark fixed objects
(which have finalizers).

This GC uses classical algorithms {\small (inspired by some A.Appel's
papers and R.Lins \& R.Jones' book)}. But I know of no other
(classical) copying generational GC, usable for C, which can be
customized to arbitrary data structures (while following stringent
coding rules). This GC is probably best suited for generated C code,
since the generator could be designed to follow every required coding
rules. A code generator might use \verb+C--+ (and \verb+QuickC--+ see
\url{http://www.cminusminus.org/}) when it will be available.

This GC is a {\em copying precise} (or exact) garbage collector. This
means that pointers are changed by the GC (which may be called at
every allocation point). The GC may change any (GC-managed) pointer on
the call stack and in the heap. So you have to particularily be
careful to tell the GC where are each pointer on the
stack\footnote{You tell the GC where are the stack pointers with the
  {\tt BEGIN\_LOCAL\_FRAME} or equivalent macro, which tells the GC
  the number of local pointers, the first of them, and the number of
  argument pointers and the first such argument.} and inside each
object in the heap\footnote{You tell the GC where are pointers inside
  heap objects by explicitly providing {\em mandatory routines} (see
  section \ref{sbsec:mandatory}) to scan and copy such heap objects.},
and the GC might change any of them (to the address of a fresh copy of
the pointed object). And the compiler should not
optimize\footnote{This is why the {\tt \bf volatile} keyword is
  required for arguments and for the {\tt \_locals\_} structure.}  too
much (eg by putting a local variable only in a register where the GC
can't change it). 

Qish should be portable to any Unix, provided that the C compiler
understands the {\tt\bf volatile} keyword as meaning that the declared
volatile pointer may be changed by the GC. Purists says that Qish does
not target the full C language\footnote{But every GC, even
  conservative, for C requires some coding style, usually much less
  restrictive than for Qish...}, but only a subset; this is true in the
sense that specific coding style is required, and that the compiler
should understand the {\tt\bf volatile} keyword as above.

 An alternative compiler on Linux
is {\tt tcc}\footnote{Tcc don't optimise much and ignores the {\tt \bf
    volatile} keyword. The generated code is about 30\% slower than
  GCC with optimisation, but the compilation time is sometimes ten
  times faster than with {\tt gcc}.} on \url{http://tinycc.org} which
compiles extremely quickly (and is very suitable for dynamically
generated C code using Qish and following Qish coding conventions).
Copying collectors {\bf condemn} a region of memory then push all
objects out of this region by copying them, forwarding the pointer to
them, and changing all pointers to the new copy.


This GC is a {\em generational} garbage collector. This means that the
garbage collector focus work on newer objects, assuming that they are
mostly temporary and will die soon. This suggests or favors a
programming style with lots of (usually temporary) object
allocations. So object allocation is quite fast on the average. But
the garbage collector has to notice and scan explicitly any old object
which has been updated by the application to point to a newer
object. This require a write barrier, i.e. the notification of object
updates thru {\tt qish\_write\_notify}.

Any precise garbage collector requires some coding conventions in C
(to know about pointers on the call stack, or for the write
barrier,...): examples include the Ocaml runtime primitives coding
conventions (see
\url{http://caml.inria.fr/ocaml/htmlman/manual032.html} section 18.5)
or the Xemacs primitives coding conventions (see
\url{http://www.xemacs.org/Documentation/21.5/html/internals_12.html}).

\newpage
\section{Short overview}

Qish contains a reusable (i.e. rather generic) generational copying
garbage collector usable from C. This garbage collector requires a
particular (and low-level) coding style.


If this stuff is useful to you, be nice to send me an email to
\mailto{basile@starynkevitch.net}. The license of this software is the
GNU Lesser General Public License \footnote{it used to be GPL only.}
(i.e. LGPL - see the COPYING file); so this is a free or opensource
software.{\scriptsize (if this license annoys you, send me an email)}.


Why the name Qish? Qish was the father of Sa�l, see 1 Samuel, chap 9
(the Bible). I'm bored of finding useful acronyms (especially pleasant
in several European languages), so I am using names from the Holy
Bible. At least I hope that Qish is not offensive\footnote{In the past
I experimented on a reflexive system that I also named Qish - they have
no much in common except the name and the author}.


\section{Requirements}

To use this package, you need the following stuff:

\begin{itemize}

\item Ruby (for scripting) see \url{http://www.ruby-lang.org/}. I
  don't know much Perl and prefer Ruby for scripting tasks


\item a Glibc system such as GNU/Linux, etc... I am using a Linux
  Debian/Sid system with a 2.4.22 kernel.

\item The GNU make utility, since I depend upon GNU make extensions -
  see \url{http://www.gnu.org/software/make/}. I am using make 3.80.
  
\item The GNU GCC compiler version 3.3 or later (I recommend against
  using 3.2, 3.1, 3.0 or 2.x versions) - see
  \url{http://www.gnu.org/software/gcc/}.
  
\item the Tiny CC compiler (by Fabrice Bellard) on
  \url{http://tinycc.org} is very useful to compile (perhaps
  generated) C code using Qish (but you may compile the runtime with
  GCC using optimisations). {\tt tcc 0.9.14} compiles very quickly
  (sometimes 10 times faster than gcc 3.2) while producing code no
  more than 30\% slower than {\tt gcc -O3}.

\item This documentation is processed with LaTeX and HeVeA. HeVeA is a good
  LaTeX to HTML translator. See \url
  {http://para.inria.fr/~maranget/hevea/} It is written in Ocaml. See
  \url{http://www.ocaml.org/} 

\item I am using PRCS for version control. See
  \url{http://prcs.sourceforge.net/}. But it is only needed in the
  {\tt \verb+scripts/qish_snap+} Ruby script. You should not need it
  otherwise.

\end{itemize}


\section{The garbage collector}

A garbage collector (or GC) manage resources (mostly memory). If you
are not familiar with garbage collection, see R.Jones's GC page on
\url{http://www.cs.ukc.ac.uk/people/staff/rej/gc.html} and also
\url{http://www.memorymanagement.org}. In this
document, an object is just a memory zone managed by the GC; it may be
or not an object visible to your application. All pointers to this
object points to the start of the object and are managed by the GC. A
word may contain a pointer or some kind of integer. On x86, words are
4 byte long.

A {\em copying} GC moves objects. The advantage of moving objects is
that freeing dead objects is easy; alive objects are moved outside a
big zone, and then this big zone is freed at once. So dead objects are
not reclaimed one by one. Also, a copying GC do compact the used
memory (but need temporarily twice as much as memory), thus avoiding
fragmentation. So, {\em pointers are changed} by the copying
GC. Therefore, all object pointers are {\tt volatile} for the C
compiler, and the garbage collector needs to know {\em about every
(garbage collected) pointer}.

A {\em generational} GC favors young objects. It separate objects in
old and new regions, and do two kinds of garbage collections (minor
and full). Objects are linearly allocated in a birth zone. When this
zone is full, a {\em minor garbage collection} is trigerred, which
copy alive born object into the {\em old} region. Once in a while a
{\em full garbage collection} is done, by condemning the previous old
region and copying alive old objects into a fresher old region (and
also working on the birth zone like the minor GC does). Care should be
taken about pointers from object in the old region to the new birth
one.  So every object modification (when a pointer field in an object
is changed) should be explicitly notified to the GC. Allocation in our
GC is much faster (in the usual case, when no garbage collection is
needed) than a cheap call to a {\tt malloc} like routine.

Our GC also provides {\em finalized fixed objects}. Such objects are
not moved, and are explicitly destroyed (one by one) by the GC, which
calls a finalization routine. Finalized objects are more costly than
copied ones. They are intended to manage external system resources
like files or windows. The finalization routine should not use or
change garbage collected pointers or fields (except by clearing
them). Our finalized objects are not like Java's in that respect: the
finalizer routine cannot allocate any GC-ed object!

Our GC also supports tagged integers. Any pointer word ending with a
set LSB bit is assumed to be such an integer. Use the {\tt
\verb+qish_is_tagged_int+} inlined function to test if a pointer is such a
tagged integer. If it is one, you can convert it to {\tt int} or {\tt
long} with the {\tt \verb+QISH_TAGGED2INT+} and {\tt
\verb+QISH_TAGGED2LONG+} macros. To make a tagged integer (ie to encode integers in a GC-ed pointer) use the {\tt \verb+QISH_INT2TAGGED+}
or {\tt \verb+QISH_LONG2TAGGED+} macros.

Qish provides global constant pointers (up to 65536 pointers). Use
{\tt QISH\_GLOBCONST({\it N})} to get the {\it N}-th constant, and
{\tt QISH\_SET\_GLOBCONST({\it N,V})} to set the {\it N}-th constant
to the pointer value {\it V}.

Qish also provides module constant garbage collected pointers (one
pointer per module).


\subsection{Introductory examples}

Notice: examples talk about Ruko which is obsolete. 

To give a concrete feeling about Qish runtime, here are some
illustrative examples from Ruko. Ruko provides (among other types)
vectors and tuples, which are sequences of garbage-collected object
pointers. Vectors are mutable (the components {\tt vect->tab} can be
changed) but tuples are immutable (the components are set at creation
time and then are read-only).

The file {\tt ruko/ruko.h} declare the following structure (common to
vectors and tuples):

\begin{quote}
{\tt% 
struct ktuple\_t \{ \\
  unsigned header; // {\it first word is a common discriminating header}\\
  void* tab[0]; // {\it next words are GC-ed object pointers}\\
\};
}
\end{quote}

The {\tt header} word encodes both a kind (which is {\tt KIND\_TUPLE}
for a tuple and {\tt KIND\_VECTOR} for a vector) and a size - which
for vectors and tuples is the number of components, i.e. the real
length of the {\tt vect->tab} field. But other uses of Qish have their
first word containing a garbage-collected {\em non-null} pointer to a
class object.

Here is the commented code (from file {\tt ruko/ruko.c}) of the {\tt
ruk\_dup\_vector} function, which duplicate a source tuple or vector
as a fresh newly allocated vector, initialized with the same
components as those of the source.  This function returns a garbage
collected pointer:

\begin{quote}
{\tt%
struct ktuple\_t*\\
ruk\_dup\_vector(struct ktuple\_t* volatile vec) \{ // {\it {\bf volatile} arguments!}
}
\end{quote}

It is mandatory that all garbage collected argument pointers should be
declared volatile as above, because the GC may move (i.e. change) such
pointers. Also, all garbage collected argument pointers should be
consecutive.

\begin{quote}
{\tt%
  struct ktuple\_t* volatile tup=0; \\
// {\it local variable {\tt tup} is also {\bf  volatile}}\\
  int kd=0; int len=0; int i=0; }
\end{quote}

We have one single local garbage collected pointer {\tt tup} (which
will hold the result of the function). Since it is alone we just
declare it volatile and initialize it to 0. If we had several local
GC-ed pointers (or even one of them) we would pack them inside a
volatile structure, conventionally named {\tt \_locals\_} and also
initialized to all 0. It is important that all local GC-ed pointers
variables are volatile, consecutive, and initialized to 0 (or a valid
GC-ed pointer). It is a good habit to initialize every local
variables, even the plain {\tt int} ones.

\begin{quote}
{\tt%
// {\it mandatory start of  GC-ed frame}\\
  BEGIN\_SIMPLE\_FRAME(1, vec, 1, tup); \\
  // {\it we have 1 garbage collected argument starting at} vec\\
  // {\it we have 1 GC-ed local pointer starting at} tup\\
}
\end{quote}

The call to the (deprecated {\tt BEGIN\_SIMPLE\_FRAME} macro, or)  {\tt
BEGIN\_LOCAL\_FRAME} when using the {\tt \_locals\_} structure) is
mandatory. It indicates to the GC the number of garbage collected
pointer arguments, and the first such argument, and the number of
garbage collected local pointers, and the first such pointer. The
expanded C code registers a GC frame (in a linked-list of frames
handled by the GC) and executes in a small constant time. The {\tt
BEGIN\_SIMPLE\_FRAME} macro call should be the first executable
statement (after initialization of variables to constants such as
pointer 0) of the body of any function using the GC.


Actually it is strongly suggested to always use a {\tt \_locals\_}
structure declared {\tt volatile struct} for local garbage collected
pointers. Actually do not use the deprecated {\tt
  BEGIN\_SIMPLE\_FRAME} macro but only use the recommended {\tt
  BEGIN\_LOCAL\_FRAME} (when you have pointers both in arguments and
in {\tt \_locals\_}), or  {\tt \verb+BEGIN_LOCAL_FRAME_WITHOUT_ARGS+}
(when you have pointers only in  {\tt \_locals\_}) or
{\tt\verb+BEGIN_FRAME_WITHOUT_LOCALS+} (when you have only pointers in
arguments).


Then we compute the kind of the source vector. We only duplicate
vectors and tuples.  To duplicate, we compute the length of the source
(taken from its header).

\begin{quote}
{\tt%
  kd = ruk\_kind(vec);\\
  if (kd == KIND\_VECTOR || kd == KIND\_TUPLE) \{\\
    len = HEADERSIZE(vec->header);
}%
\end{quote}

Once the length is computed, we allocate the resulting {\tt tup} with
a call to {\tt qish\_allocate}. This call may trigger a garbage
collection, which may change (by moving them) some or even all garbage
collected pointers, including the current frame local pointers or
argument pointers. With compilers without inlining, you might
consider using the {\tt QISH\_ALLOCATE} macro (invoke it without any
side-effects in arguments, eg {\tt\verb!QISH_ALLOCATE(--p,sz++);!} is
incorrect). 

More generally, you inhibit all small inline functions by
compiling your application with the {\tt\verb+-DNO_QISH_INLINE+}
flag. But then you have to use macros instead of functions.

\begin{quote}
{\tt%
    tup = qish\_allocate(sizeof(*tup) + len*sizeof(void*));\\
    //{\it a garbage collection may occur above, changing many pointers}
}%
\end{quote}

The {\tt qish\_allocate} function returns a zeroed chunk of memory. We
have to initialize it by filling all relevant fields:

\begin{quote}
{\tt%
    tup->header = MAKEHEADER(KIND\_VECTOR, len);\\
    for (i=0; i<len; i++) tup->tab[i] = vec->tab[i];\\
  \} // end if kind is KIND\_VECTOR or KIND\_TUPLE
}%
\end{quote}

Each function should end with the macro call to {\tt EXIT\_FRAME},
which pops the frame registered by (the deprecated {\tt BEGIN\_SIMPLE\_FRAME}) or {\tt
BEGIN\_LOCAL\_FRAME}. This macro should only be followed by a simple
{\tt return} statement, which returns a constant or a simple variable:

\begin{quote}
{\tt%
  EXIT\_FRAME();\\
  return tup;\\
\} // end of ruk\_dup\_vector
}%
\end{quote}

The preferred coding style is to have a {\tt \_locals\_} structure, which
should be {\bf volatile} and initially cleared!

\begin{quote}
\tt {\it // preferred coding}\\
void* foo(struct ktuple\_t* volatile tup, \\
   \hspace*{2.5cm} void* volatile val) \{\\
{\it // declare and clear a volatile \_locals\_ structure for local pointers}\\
 volatile struct \{ void* ptr; void* res; \} \_locals\_ = \{0,0\};\\
 BEGIN\_LOCAL\_FRAME(2,tup);\\
 {\it //... use \_locals\_ fields as the only GC pointer local variables}\\
 if (some\_condition(tup)) \\
   \_locals\_.res = val;\\
 {\it //...}\\
 EXIT\_FRAME();\\
 return \_locals\_.res;\\
\}
\end{quote}

This idiom is so common that a tiny Ruby script {\tt scripts/gen\_locals}
exist to generate macro definitions like 
\begin{quote}
\tt%
{\it // generated by gen\_locals script} \\
{\#}define l\_res \_locals\_.res
\end{quote}

The script generate such definitions for every different occurrence of
{\tt l\_*} names in the source. So just code {\tt l\_res}, run the
script (with the {\tt .c} source as first argument and the generated
{\tt .h} as second argument), and include its output near the start of
your file.

\smallskip

A function which changes a garbage collected object by updating a
pointer field in it should notify the garbage collector, for example:

\begin{quote}
{\tt%
// {\it this function set the component of a vector returning its previous value}\\
void*\\
ruk\_vector\_set(struct ktuple\_t* volatile tup, void* volatile val, int rk) \{\\
  int sz=0;\\
  void* oldval=0;\\
  BEGIN\_SIMPLE\_FRAME (2, tup, 1, oldval);\\
  if (ruk\_kind(tup) != KIND\_VECTOR) goto end;\\
  sz = HEADERSIZE(tup->header);\\
  if (rk<0 || rk>=sz) goto end;\\
  oldval = tup->tab[rk];\\
  tup->tab[rk] = val;\\
  qish\_write\_notify(tup); // {\it {\bf notification} of a changed object}\\
 end:\\
  EXIT\_FRAME();\\
  return oldval;\\
\} 
}
\end{quote}

Note that every function which may use directly or indirectly the GC
should follow the same coding rules (detailed below). Using the GC
means either allocating new objects or calling functions which may use
the GC. Following these rules is ok even for other functions, when in
doubt always follow them.

The macro {\tt\verb+QISH_WRITE_NOTIFY+} is equivalent to the {\tt
  qish\_write\_notify} function, provided this macro is invoked
  without side-effects.


The application has to follow some (very liberal) rules regarding data
structures.

\subsection{Data structure of objects}
\label{sec:datastruct}

Moved objects should all start with a common prefix (i.e. a
word). {\em Their first word should never be zero}\footnote{A zeroed first
word indicates forwarded objects to the GC}, so it can be a non-nil
pointer \footnote{The pointer could be some kind of descriptor, or
even a C++ vtable pointer, if you have a tree of classes sharing a
common root class, with only single inheritance, and virtual methods.}
(garbage collected or not) or a header word.  All objects should be at
least two words long. The GC do not need any additional word for moved
objects.


Objects should know their size (the GC do not manage by itself the
objects' size) and their data type.

For example, one could start every object with an {\tt unsigned
header} whose topmost byte is a non-zero kind number and whost 3
lower bytes encode some sizing information (dependent on the kind).

Your application has to define all its objects types and data
representations, provided they start with a never-zero word or pointer.

As a simple toy example (nearly meaningless, only for illustrative
purposes), suppose you are coding an integer calculator with formal
variables. Then each object can start with two half-words, a kind and
a size or code. Objects can be binary-operations, or variables (or
tagged integer).

\begin{verbatim}
enum {
  KIND_NONE=0 /*unused*/,
  KIND_BINOP,
  KIND_VARIABLE
};
\end{verbatim}


Binary operations in your calculator will be represented like:

\begin{verbatim}
  enum { OP_NONE, OP_ADD, OP_SUB, OP_MULT, OP_DIV };

  struct binop_st {
    short kind; /*always KIND_BINOP*/
    short opcode;
    void* left;
    void* right;
  };
\end{verbatim}


Variables have a value (either a tagged integer or an binary
operation) and a name; the size is the length of the name, and
variables are objects of various size.

\begin{verbatim} 
  struct variable_st {
    short kind; /*always KIND_VARIABLE*/
    short namelen; /*length of name*/
    void* value; /*value of variable*/
    char name[1]; /*actually [namelen] bytes + final '\0' */
  };
\end{verbatim}


\subsection{required coding practices}

Our GC is not conservative\footnote{Conservative GCs [e.g. Boehm's]
are much easier to use, but they might leak, may be slower on some
applications, and do not compact memory. But Boehm's GC is not
disruptive like Qish, and has an API compatible with (or similar to)
{\tt malloc}.}, but exact. It has to know about every garbage
collected pointer, and usually modify them (when copying alive
objects).


\subsubsection{No interior pointers}

It is not allowed to have pointers to the inside of any garbage
collected object. Every pointer should point to the start of such
objects. {\small (therefore, your C++ application cannot have multiple
inheritance)}

\subsubsection{No multi-threading}

Our GC does not support threading. If you dare use Posix threads, be
careful that only one thread should use the GC.

\subsubsection{Limited global data}

You should have almost no global pointers (I believe having lots of
global data is a bad practice). The only permissible exceptions are:

\begin{enumerate}

\item you can use the small (fixed size) 
%
\footnote{
  The global roots is an array of {\tt QISH\_NB\_ROOTS}
  pointers defined as 64 in {\tt qish.h}. You could if needed change it
  to a rather small value (at most a few hundreds): at every garbage
  collection, the whole roots array is scanned and updated, so having
  lots (e.g. thousands) of such roots will not be reasonable.
}
%
array of global pointers {\tt \verb+qish_roots+} as you wish, reading
and writing in it any pointer to a GC-ed object (or the nil pointer,
or a tagged integer).

\item Each module is described by an entry in {\tt
  \verb+qish_moduletab+}. Each such entry contain a rather constant
  pointer. You can set or change the constant of module $�$ to pointer
  $p$ by calling {\tt \verb+qish_changeconstant+($i$, $p$)}, and you
  can get this constant by {\tt \verb+qish_constant+($i$)} or even accessing
  directly the {\tt \verb+km_constant+} field in the entry of {\tt
  \verb+qish_moduletab+}. In practice, you could make this constant
  point to a structure of GC-ed pointers. 

\end{enumerate}

\subsubsection{Pointers are volatile}

All pointers are volatile (because the GC silently moves them), in
particular arguments should be volatile. You should compile with the
{\tt -fvolatile} and {\tt -fvolatile-global} flags to the {\tt gcc} compiler.
Most importantly, you should explicitly declare volatile your
formal arguments. So the following is incorrect

\begin{verbatim}
#error formal argument not declared volatile
void* foo(struct variable_st* var) { /*body*/ }
\end{verbatim}

You should definitely code instead like this - notice that the {\tt
  volatile} qualifier goes {\em after} the \verb+*+ indicating a pointer):

\begin{verbatim}
/* explicit volatile argument */
void* foo(struct variable_st* volatile var) { /*body*/ }
\end{verbatim}

Omitting the {\tt volatile} qualifier does produce hard to find bugs.

\subsubsection{Indicate arguments and locals to the GC}

Each function body
\footnote{At least each function using directly or
  not the GC, i.e. either allocating memory with {\tt qish\_allocate}
  or calling -directly or indirectly- any function which itself does
  allocation} 
%
should start with a prologue and end with an epilogue. The prologue
 mark the current call frame (remembering the first argument, the
 first local pointer, and the number of arguments and of local
 pointers), and the epilogue reset the previous call frame.  The
 prologue is \\
 {\tt\verb+BEGIN_SIMPLE_FRAME+($nbparam$,$firstparam$,$nblocal$,$firstlocal$)};\\
 the epilogue is {\tt \verb+EXIT_FRAME+();}. The cpu time cost of the
 prologue or epilogue small, nearly constant, and independent of the
 numbers of locals or arguments. Actually, this
 {\tt\verb+BEGIN_SIMPLE_FRAME+} macro is deprecated. Use the
 {\tt\verb+BEGIN_LOCAL_FRAME+} macro (when you have parameters and a
 {\tt \verb+_local_+} struct), or
 {\tt\verb+BEGIN_FRAME_WITHOUT_LOCALS+} macro (when you have
 parameters but no local pointer), or
 {\tt\verb+BEGIN_LOCAL_FRAME_WITHOUT_ARGS+} macro (when you have a
 {\tt \verb+_local_+} struct without any pointer parameters).


All local pointers should be explicitly initialized to 0 (or a simple
value).

It is not permissible to return from a function without going thru
{\tt \verb+EXIT_FRAME+(); } the suggested coding convention is the
have one single exit point, ie a single {\tt \verb+EXIT_FRAME+(); } at
the end followed by a {\tt return} statement. If a function returns a
garbage-collected value, it should be a local pointer.


For convenience, there is also a
{\tt\verb+BEGIN_LOCAL_FRAME+($nbparam$,$firstparam$)} macro, which
assumes that a local variable named {\tt \_locals\_} is defined as
a structure containing only garbage-collected pointers. A {\it Ruby}
script {\tt gen\_locals} is provided to generate (in a separate file
to be included) for each variable named like {\tt l\_*}, e.g. {\tt
l\_foo} a macro {\tt \verb+#define l_foo _locals_.foo+}; in practice,
local GC-ed pointers eg {\tt bar} should be declared as a pointer
field in {\tt \_locals\_} and referred as {\tt l\_bar}.

For convenience, there is a 
{\tt\verb+BEGIN_FRAME_WITHOUT_LOCALS+($nbparam$,$firstparam$)} macro
to be used when you don't have any local garbage-collected pointer
(but only parameters). Symetrically, a 
{\tt\verb+BEGIN_LOCAL_FRAME_WITHOUT_ARGS()+} macro is provided, when
you only have local garbage collected pointers inside your usual {\tt
  \_locals\_} structure.


If a frame (therefore a function body) have no local pointers (or no
pointer arguments) it can pass {\tt qish\_nil} (or any unused adress)
as the appropriate argument to the {\tt BEGIN\_SIMPLE\_FRAME} or {\tt
  BEGIN\_LOCAL\_FRAME} macro. For example, a function without
arguments and with {\tt \_locals\_} encapsulated pointer variable
should start with a {\tt BEGIN\_LOCAL\_FRAME(0,
  qish\_nil)}. Internally {\tt qish\_nil} is a pseudo-data whose
address is the nil pointer.

Every local GC pointer should be explicitly initialized to a simple
value (usually the null pointer). It is best to clear any pointers
heavily used in a loop when exiting out of the loop.

You cannot use {\tt longjmp} without special measures. If you want exceptions,
use the {\tt BEGIN\_EXCEPT\_BLOCK CATCH\_EXCEPT\_BLOCK  END\_EXCEPT\_BLOCK
  THROW\_EXCEPTION} macros from file {\tt qish.h}


\subsubsection{No composition of function calls}

Since each pointer should be known to the GC (either as a argument or
a local explicited thru {\tt\verb+BEGIN_SIMPLE_FRAME+} or
{\tt\verb+BEGIN_LOCAL_FRAME+}, or as a global root or constant, or as
a field in a garbage collected object) it is not permissible to call
several functions, e.g.

\begin{verbatim}
#error no function composition
l_res = f(g(y),l_z->ptrfield);
\end{verbatim}

but you should code thru a temporary value
\begin{verbatim}
/* use a temporary variable */
l_tmp = g(y);
l_res = f(l_tmp, l_z->ptrfield);
}
\end{verbatim}

\subsubsection{Complete allocation}

Every allocation of a garbage collected object is done thru a call to
one of the following functions (which may trigger a garbage
collection): 

\begin{itemize}

\item {\tt qish\_allocate($bytesize$)} to allocate an ordinary (movable)
  object with a natural (word) alignement. This is the most often used
  allocation function; usually the $bytesize$ is some {\tt sizeof(type)}.

\item {\tt qish\_allocate\_aligned($bytesize$,$alignment$)} to allocate
  a movable object with an explicit $alignment$ expressed in bytes
  (which must be a small power of 2 in words).

\item {\tt qish\_fixed\_alloc($bytesize$,$finalizer$)} to allocate a
  fixed finalized object (aligned to at least the {\tt
  sizeof(double)}), with an optional finalising routine (called by the
  GC with the adress of the fixed object).

\end{itemize}

Once an object is allocated it can be (and should be) filled. The
allocated object should be filled to become valid (for marking and
scanning routines) before the next allocation. Memory provided by the
above allocation routines is cleared to all-zero bytes.

It should be noted that in the usual case {\tt qish\_allocate} and
{\tt \_allocate\_aligned} are very quick and inlined routines
(basically a pointer increment and a compare to the birth region
limit) which occasionally triggers a garbage collection. Adventurous
expert users could even allocate several object at one with a single
call to {\tt qish\_allocate}, giving it the cumulated total size of
all allocated objects.

Since object allocation is very quick (much faster than a call to {\tt
malloc}) it is expected that the application makes frequent
allocation to short lived objects.

The macro {\tt \verb+QISH_ALLOCATE+} does the same as {\tt
  qish\_allocate} provided its invocation has no side-effects.

\subsubsection{Notification of updates (write barrier)}

Since the garbage collector has to track pointers from old to new
generation, it should be aware of any updates of allocated
objects. This is done by calling {\tt qish\_write\_notify($objptr$)}
after changing the GC-ed pointers in the object $objptr$ and before
any further allocation or call.

It is not required to notify the GC after any non-pointer updates
(e.g. adding bytes into a garbage collected string). It is better to
code some useless calls to {\tt qish\_write\_notify} than to forgot
one. 

A garbage collection can be triggered at each {\tt
qish\_write\_notify} points.


Explicit update notification favors a functional programming style
(where updates are rare).

Assignement to local GC pointers, to arguments, and root variables do
not require any notification. 

\subsubsection{Optional explicit garbage collection}

The garbage collector can be explicitly called by the application with
\\ {\tt qish\_garbagecollect($size$,$fullflag$)} where the $size$ is
an estimation (which can be left as 0) of the needed size -in bytes-
of future objects and the $fullflag$ is non-zero to force a full
garbage collection (otherwise, a minor collection will be done, unless
the old generation has grown significantly).

To be sure that at least $size$ bytes are allocatable without GC, you
can call {\tt qish\_reserve($size$)} which calls the garbage collector
unless $size$ bytes are available in the birth zone. This is
particularily useful in applications having garbage-collected type
descriptors, which have to bootstrap and fill the type descriptors'
descriptor without any garbage collection.

It could be interesting to trigger an explicit garbage collection once
in a while in an idle loop, or before an important processing
requiring lots of allocation or recursion. This is always an
optimisation, and the GC will work without a single explicit call to
{\tt qish\_garbagecollect} in the application code.
 

\subsubsection{Exception handling}

Exception handling has to cooperate with the GC, because of the frame
linking mechanism. You cannot simply call {\tt longjmp} or throw C++
exceptions\footnote{If you use C++ exceptions, ensure (by coding
  tricky appropriate constructors and destructors) that the {\tt
    EXIT\_FRAME} is called on exceptional frame unwinding. You are on
  your own.}. Instead you have first to declare (in your application)
one (or very few) {\tt void*} global (or static) pointer variable e.g.
{\tt exv}. You also need a local integer variable {\tt cod} holding a
non-zero error or exception code (usable as you want, but never 0 if
exception thrown, it is the result of {\tt setjmp}) and a local
pointer {\tt\verb+_locals_.exobj+} holding any exception
(garbage-collected) object. 

Then you surround any potentially exception-raising code
(after the usual {\tt BEGIN\_LOCAL\_FRAME}... or similar macro)
with:

\begin{quote}
\tt\verb+BEGIN_EXCEPT_BLOCK(exv);+\\
\it // your code here may directly or indirectly throw ``exceptions''
\end{quote}

Then you code as usual, you can call some other routines which do
allocation and may indirectly throw an ``exception'' (using the
{\tt\verb+THROW_EXCEPTION+} macro detailed below). The {\tt exv}
argument to the {\tt\verb+BEGIN_EXCEPT_BLOCK(exv);+} macro (vaguely
similar to the {\tt try} keyword of Java or Ocaml) holds the
adress of the exception-catching frame, and this
{\tt\verb+BEGIN_EXCEPT_BLOCK(exv);+} opens a brace (so starts a C code
block). Then you catch exceptions with 

\begin{quote}
\tt\verb+CATCH_EXCEPT_BLOCK(cod,_locals_.exobj);+\\
\it // your code handle here exception of error {\tt cod} ...\\
\it // ... with {\tt\verb+_locals_.exobj+} set to the exception object
\end{quote}

So this {\tt\verb+CATCH_EXCEPT_BLOCK+} is vaguely similar to the {\tt
  with} keyword of Java or Ocaml exception handlers. At last you have
  to end the exception hanling code with
\begin{quote}
\it // this ends the exception handling code\\
\tt\verb+END_EXCEPT_BLOCK(exv);+
\end{quote}

The {\tt void*} variable argument to
{\tt\verb+END_EXCEPT_BLOCK(exv);+} should always be the same as the
argument of the matching previous
{\tt\verb+BEGIN_EXCEPT_BLOCK(exv);+}.

Inside the normal block enclosed with {\tt\verb+BEGIN_EXCEPT_BLOCK+}
and {\tt\verb+CATCH_EXCEPT_BLOCK+} you can call functions (or even
directly) which throws (directly or indirectly) an ``exception'' with
{\tt\verb+THROW_EXCEPTION(exv,+}$Cod$ {\tt ,} $Exob$ {\tt )}. The
$Cod$ should be a non-zero integer (an error code, passed as the
second argument to {\tt longjmp}) and the $Exob$ is the
(garbage-collected) error object. If this ``exception'' throwing
happens, control jumps to the {\tt\verb+CATCH_EXCEPT_BLOCK+}.

Of course, the {\tt\verb+BEGIN_EXCEPT_BLOCK+},
{\tt\verb+CATCH_EXCEPT_BLOCK+} and {\tt\verb+END_EXCEPT_BLOCK+} have
to be in the same C code block (so the same function).


\subsubsection{Utility routines}
\label{sbsec:utility}

The Qish runtime provide some utility functions, in particular:

\begin{itemize}

\item {\tt qish\_strhash($string$,$length$)} compute an hashcode of a
  given $string$ with an explicit $length$; if this length is
  negative, the $string$ is supposed null-terminated, as if $length ==
  strlen(string)$ 

\item{\tt qish\_sigexecvp($file$,$argv$)} spawn a process to execute
  $file$ with the given program $argv$ null-terminated arguments and
  wait for its completion.

\item {\tt qish\_prime\_after($i$)} returns a prime number bigger than
  $i$ provided that $0 < i < 10000000 = 10^7$ which can be useful for
  hashtables, etc.

\item {\tt qish\_parameter($name$)} retrieves the string-value of a
  runtime parameter of a given $name$.

\item {\tt qish\_put\_parameter($name$,$val$)} put into the runtime
  paramter named $name$ the string value $val$

\item {\tt qish\_parse\_configfile($filename$)} parse a simple
  configuration file and set parameters appropriately

\item{\tt qishgc\_init();} should be called once to initialize the
  garbage collector, before any allocation!

\item{\tt qish\_load\_module($modulename$,$rank$)} loads (with {\tt
  dlopen}) a shared object module at a given rank. It returns 0 iff
  ok. A previous module at the same rank is closed latter with {\tt
  qish\_postponed\_dlclose}

\item{\tt qish\_get\_symbol($name$,$modrank$)} gets the address of the
  symbol of given $name$ in a module of given $modrank$ or in any
  modules if $modrank < 0$

\item{\tt qish\_postponed\_dlclose()} close any previous module with
  {\tt dlclose} and should be called when the call stack is very low
  (i.e. in your event loop)

\item{\tt qish\_panic} is a printf-like macro which aborts after
  displaying a panic message.

\end{itemize}

\subsection{Mandatory routines to the GC}
\label{sbsec:mandatory}

Your application should provide four mandatory routines to the GC and
store their address in global function pointer variables before any GC
call. All such functions should be provided (give a dummy function if
not needed). Since these functions are called by the GC, they should
not follow the above coding guidelines (no {\tt BEGIN\_SIMPLE\_FRAME}
etc...) and should of course never allocate objects or call the GC.


\subsubsection{Copying function {\tt qish\_gc\_copy\_p}}

The function pointer {\tt qish\_gc\_copy\_p} should be set by your
application to a routine which copy an object (into an address provided
by the GC). Its prototype is {\tt void* $gc_{copy}$ (void**padr, void* dst, const
void* src)}. \\
It should set {\tt *padr} to the new adress of the copy
(usually {\tt dst} or some aligned word after) and should return the
first word after the copied object.

For simple cases (word aligned structures) the copy routine can be as
simple as e.g. a switch of cases like

\begin{verbatim}
/*after determining the dynamic object type*/
*((object_type*)dst) = *((object_type*)src); 
 return ((object_type*)dst)+1; 
\end{verbatim}

after having determined the dynamic object type (e.g. thru its header).

The first word of the copied object should not be zero.

\subsubsection{Minor scan of a movable object \tt  qish\_minor\_scan\_p}

The function pointer {\tt qish\_minor\_scan\_p} should be set by your
application to a routine which scans for the minor garbage collection
and object by updating each of its pointer fields with the {\tt
QISHGC\_MINOR\_UPDATE} macro (called with the pointer field). This
routine should return the next word after the scanned object. Its
prototype is {\tt void* $minor_{scan}$ (void*ptr);} and it should return the
first word after the scanned object at address {\tt ptr}.


For simple cases (word aligned structures) the minor scan routine can be as
simple as e.g. a switch of cases like

\begin{verbatim}
/*after determining the dynamic object type*/
QISHGC_MINOR_UPDATE(((object_type*)src)->ptrfield1);
QISHGC_MINOR_UPDATE(((object_type*)src)->ptrfield2);
/* etc for every field*/
 return ((object_type*)src)+1; 
\end{verbatim}

after having determined the dynamic object type (e.g. thru its header).

When you are sure that the pointer is a true pointer (ie is never a
tagged integer) you can use {\tt QISHGC\_MINOR\_PTR\_UPDATE} instead
of {\tt QISHGC\_MINOR\_UPDATE}. 

Qish accepts not only null pointers, but also any address inside the
first page of address space (so on x86 any address below 0x1000). For
instance you could mark emptied slots in hashtable specially by such
an address (eg {\tt (void*)16}).

\subsubsection{Full scan of a movable object \tt  qish\_full\_scan\_p}

The function pointer {\tt qish\_full\_scan\_p} should be set by your
application to a routine which scans for the full garbage collection
and object by updating each of its pointer fields with the {\tt
QISHGC\_FULL\_UPDATE} macro (called with the pointer field). This
routine should return the next word after the scanned object. Its
prototype is {\tt void* $full_{scan}$ (void*ptr);} and it should return the
first word after the scanned object at address {\tt ptr}.


For simple cases (word aligned structures) the full scan routine can be as
simple as e.g. a switch of cases like

\begin{verbatim}
/*after determining the dynamic object type*/
QISHGC_FULL_UPDATE(((object_type*)src)->ptrfield1);
QISHGC_FULL_UPDATE(((object_type*)src)->ptrfield2);
/* etc for every field*/
 return ((object_type*)src)+1; 
\end{verbatim}

after having determined the dynamic object type (e.g. thru its
header).

When you are sure that the pointer is a true pointer (ie is never a
tagged integer) you can use {\tt QISHGC\_FULL\_PTR\_UPDATE} instead
of {\tt QISHGC\_FULL\_UPDATE}.

Usually the full scanner has the same code as the minor scanner,
except for the update macros {\tt QISHGC\_FULL\_UPDATE} instead of
{\tt QISHGC\_MINOR\_UPDATE}.

If the first word of your object is not a header but a garbage
collected pointer -for instance the class pointer in a ObjVlisp-like
object language (i.e. single inheritance language with classes reified
as objects)-, be careful in your scanning GC routines to
check that it is a pointer (you could use the
{\tt\verb+QISH_IS_MOVING_PTR+(}$ptr${\tt )} macro), then update the
pointer first (with {\tt QISHGC\_FULL\_UPDATE} or {\tt
  QISHGC\_MINOR\_UPDATE}), then use it appropriately (your reified
class could contain a description of its fields, or scanning routine
pointers, ...).


\subsubsection{Full scan of a fixed object \tt  qish\_fixed\_scan\_p}

The function pointer {\tt qish\_fixed\_scan\_p} should be set by your
application to a routine which scans for the full garbage collection
and object by updating each of its pointer fields with the {\tt
QISHGC\_FULL\_UPDATE} macro (called with the pointer field). This
routine returns void and knows about the fixed object size. Its
prototype is {\tt void $fixed_{scan}$ (void*ptr, int size);} where
{\tt ptr} is the address of the object and {\tt size} is its size in
bytes.


For simple cases (word aligned structures) the fixed scan routine can be as
simple as e.g. a switch of cases like

\begin{verbatim}
/*after determining the dynamic object type*/
QISHGC_FULL_UPDATE(((object_type*)src)->ptrfield1);
QISHGC_FULL_UPDATE(((object_type*)src)->ptrfield2);
/* etc for every field*/
 return;
\end{verbatim}

after having determined the dynamic object type (e.g. thru its
header).

Even an application with only moving objects (and no finalized, fixed
objects) should provide a dummy fixed scanner.

For the simple examples in section \ref{sec:datastruct}, the routines
could be:

\begin{verbatim}
void* gc_copy(void**padr, void* dst, const void* src) {
  switch (*(short*)src) {
  case KIND_BINOP:
    *((struct binop_st*)dst) = *((struct binop_st*)src);
    return ((struct binop_st*)dst)+1;
  case KIND_VARIABLE:
    {  struct variable_st* srcvar = src;
       int srcnamlen = srcvar->namelen;
       memcpy(dst, src, sizeof(struct variable_st)+srcnamlen);
       if (srcnamlen & (sizeof(void*)-1)) {
         /*round up name length to word*/
         srcnamlen |= (sizeof(void*)-1); srcnamelen++;
       }
       return ((struct variable_st*)dst)->name + srcnamlen;
    }
  }
}

void* minor_scan(void*ptr) {
  switch (*(short*)ptr) {
  case KIND_BINOP: 
    QISHGC_MINOR_UPDATE(((struct binop_st*)ptr)->left);
    QISHGC_MINOR_UPDATE(((struct binop_st*)ptr)->right);
    return ((struct binop_st*)ptr)+1;
  case KIND_VARIABLE:    
    {  struct variable_st* var = ptr;
       int namlen = var->namelen;
       QISHGC_MINOR_UPDATE(var->value);
       if (namlen & (sizeof(void*)-1)) {
         /*round up name length to word*/
         namlen |= (sizeof(void*)-1); namelen++;
       }
       return var->name + srcnamlen;
    }
  }
} 

\end{verbatim}

The {\tt full\_scan} routine would be similar (using {\tt QISHGC\_FULL\_UPDATE}).

Since there are no fixed object you have to provide a dummy fixed
scanning routine which just calls {\tt abort}

\subsection{Using Qish in C++ code}

I just give very few hints on using Qish in C++ code:

\begin{enumerate}

\item no implicit or explicit use of {\tt this}. Since it is not
  possible to declare the argument {\tt this} to be a volatile
  pointer, you should not use it (either explicitly as in {\tt
    this->method(foo)} or {\tt this->\_field} or implicitly as in {\tt
  method(foo)} or {\tt \_field}). Instead, copy {\tt this} to a local
  pointer, using it explicitly. This stylistical constraint is
  annoying.

\item no multiple inheritance. Since Qish does not support interior
  pointer, you cannot use multiple inheritance.

\item Have a tree and not a forest of classes by defining a common
  superclass to all your GC-ed objects. The virtual table pointer is
  usually the first word of such objects.

\end{enumerate}

I would suggest to avoid using C++ with Qish on new code. Instead
consider using other programming languages like Ocaml, CommonLisp,
Java.

\subsection{Advanced application explicit forwarding.}

Advanced applications using Qish (and understanding quite well its
internal processing) may explicitly forward pointers in the
application code. Use this (experimental) feature with caution. 

Explicit forwarding could be useful when you want every pointer to a
(garbage-collected and allocated) address $\alpha$ to be replaced by a
fixed $\beta$. A typical application could be to grow existing values,
dynamically change the class (hence the size) of an object, etc...

An application can for such an explicit forwarding with the macro call
{\tt QISH\_EXPLICIT\_FORWARD($alpha$,$beta$)}. Then the garbage
collector will replace any\footnote{So all occurrences are replaced
  only after a major full garbage collection; after a minor collection
  only some pointers are replaced!} pointer to $alpha$ than its scans with
$beta$.

If an application use this explicit forwarding, it has to follow
(explicitly in application code) any potential such pointer with the
{\tt QISH\_FOLLOW\_FORWARD} macro which should be applied to every
pointer (either argument, or local after assignment). If the pointer
is never a tagged integer, you may call {\tt
  QISH\_FOLLOW\_FORWARD\_PTR} directly. If you statically know that
only some pointers may have been explicitly forwarded by {\tt
  QISH\_EXPLICIT\_FORWARD} you mayy (at your own risk) call {\tt
  QISH\_FOLLOW\_FORWARD} only for such pointers.

Refer to the {\tt include/qish.h} file for definitions of these
macros.

{\em Explicit application forwarding is an experimental untested
feature}. Use it at your own risk! If you happens to use it, be kind
enough to explain me why.


\section{A tiny benchmark}
\label{sec:bench}

A tiny benchmark (adapted from the GCbench.c by H.Boehm, J.Ellics,
P.Kovac, W.Clinger, et al) is ported to qish. 


\subsection{benchmark results}
It is the file {\tt
GCBench.c} (where we changed the allocate sized to 4 times the
original) in our {\tt lib/} subdirectory (where you can run all 3
benches with {\tt make OPTIMFLAGS='-O2 -DNDEBUG' clean lib bench}. The
same file {\tt GCBench.c} compiles with Qish GC, with Boehm's GC, and
with manual malloc and free according to the setting of preprocessor
flags {\tt QISH} for Qish, {\tt GC} for Boehm, and no flags for malloc
and free. (Times have been measured with release 0.3 of Qish).

\begin{itemize}

\item Qish GC (standard birth size of 8Mbytes): CPU 9.160 user + 4.510
system = 13.670 total time (sec); done 131 minor and 16 full garbage
collections

\item Boehm's GC: CPU 18.570 user + 0.310 system = 18.880 total time
  (sec); Completed 42 collections

\item explicit malloc and free:  CPU 24.810 user + 2.810 system =
  27.620 total time (sec)

\item Qish GC with the birth size reduced to 4 Mbytes made with {\tt
  make OPTIMFLAGS='-O2 -DNDEBUG -DMIN\_BIRTH\_SIZE=4194304' clean lib
  benchqish}: CPU 13.810 user + 6.350 system = 20.160 total time
  (sec); Qish done 257 minor and 39 full garbage collections

\item Qish GC with the birth size increased to 16Mbytes: 
 CPU 6.900 user + 4.350 system = 11.250 total time (sec)
Qish done 67 minor and 6 full garbage collections

\item Qish GC with the birth size increased to 32Mbytes (probably not
  significant, since it is similar to the live object size): CPU 5.920
  user + 3.700 system = 9.620 total time (sec) Qish done 34 minor and
  2 full garbage collections

\item Boehm's GC with holes {\tt make OPTIMFLAGS='-O2 -DNDEBUG
  -DHOLES' clean lib bench}, so for every used node allocated, an
  extra useless (dead) node is also allocated: CPU 43.220 user + 0.380
  system = 43.600 total time (sec) Completed 72 collections

\item Qish GC with holes (and standard birth size of 8Mbytes): 
 CPU 12.490 user + 7.380 system = 19.870 total time (sec)
Qish done 197 minor and 23 full garbage collections

\item Qish GC with holes and increased birth size of 16Mbytes:
 CPU 9.560 user + 6.460 system = 16.020 total time (sec)
Qish done 99 minor and 10 full garbage collections

\item Qish GC with holes and small birth size of 4Mbytes: CPU 17.530
user + 8.500 system = 26.030 total time (sec) Qish done 389 minor and
52 full garbage collections

\end{itemize}

Obviously, this small benchmark does not prove much. But Qish is not
too bad, even w.r.t. the famous Boehm's (et al.) conservative (and
quite mature) garbage collector.

Explicit bug-prone manual memory management with the infamous {\tt
  free} routine is not only harder to code, but seems even slower than
  all the other automatic memory management techniques.

Qish is highly sensitive to the birth size. This is expected (since a
GC is trigerred only when the birth region is full). Qish requires lot
of system calls, because it does {\tt mmap}-ing at every GC. Perhaps
we could improve it by caching memory zones... (but the GC requires
zero-ed memory). Qish tuning can be done by carefully changing some
compile-time constants (notably {\tt MIN\_BIRTH\_SIZE MAX\_BIRTH\_SIZE
FULL\_GC\_PERIOD}) at the start of file {\tt lib/qigc.c}.

Qish works well with holes (because it compact them) and is designed
with allocation of small short lived temporary objects in mind (which
may favor some ``functional'' style of coding).


It would be very interesting to port Qish to an existing major GC-ed
application (like guile, some application using Boehm's GC, or even
emacs) but I have not enough time for this. I am willing to help any
person which wants to do so.

Qish is designed to be used for C code generators.

\subsection{Tuning Qish}

To optimize Qish for your needs you could :

\begin{itemize}

\item use the usual trick of allocating several objects simultanously
  (in one single {\tt qish\_allocate} for all of them). 
  
\item explicitly invoke {\tt qish\_garbagecollect} when needed
  (e.g. at start of a topmost loop...).

\item set the number {\tt QISH\_NB\_ROOTS} of variable roots (ie the
  size of the {\tt qish\_roots} global array) in {\tt include/qish.h}
  to your needs. Leave it at most to a few hundred or dozen.
  
\item change the {\tt MIN\_BIRTH\_SIZE} and {\tt MAX\_BIRTH\_SIZE} and
  {\tt FULL\_GC\_THRESHOLD} in {\tt lib/qigc.c}. The minimal birth
  size is a sensitive number (8 megabytes by default). I believe it
  should be at least half a megabyte (and at most a fraction, e.g. the
  tenth, of your available RAM). The maximal birth size limits the
  maximal size of allocated objects.

\item change the {\tt QISH\_MAXNBCONST} number (65536) to a power of
  two (at least 1024). It is the maximal number of global constant
  pointers (for {\tt QISH\_GLOBCONST})

\item if your plateform has enough registers (this is false for x86 with 6
  usable registers only) you could (with GCC) reserve global registers for
  {\tt\verb+qishgc_birth_cur+} and {\tt\verb+qishgc_birth_storeptr+}
  global variables (in {\tt include/qish.h}).

\end{itemize}

%%%%%%%%%%%%%%%%%%%%%%%%%%%%%%%%%%%%%%%%%%%%%%%%%%%%%%%%%%%%%%%%%%%%%%

\section{ToDo list (multi-threading?)}

Some people expressed the wish of making Qish multi-threaded (using
Posix threads ie {\tt <pthread.h>}), in the sense of having a few
threads\footnote{If you need multi-threaded capable garbage
  collection, I suggest using Hans Boehm's collector.}  concurrently
allocating garbage collected objects. This could be doable with a
``stop the world'' strategy: when a thread requires a (major)
collection it has to stop all other threads, but minor collections
eremains local to threads (and sending objects between threads
requiring some special precautions). This approach works only for a
few (at most a dozen) threads.

The problem with this approach is that the current allocation pointer
{\tt qishgc\_birth\_cur} has to become a thread-local variable. And
there is no standard mechanism providing them very quickly: I believe
that the standard {\tt pthread\_getspecific} function (which would
have to be called at each allocation with {\tt qish\_allocate} - even
those which do not trigger any collection) would significantly slow up
this runtime.

I am not very fluent with multi-threading applications, and I don't
have any biprocessor machine at home yet (experimentally a biprocessor
machine is almost required to test multithreading applications).

People having access to 64 bits machines could try to compile Qish on
these. Contributions are welcome!

Comments on this are welcome.


\end{document}
%eof $Id: qishintro.tex 19 2005-05-08 17:35:50Z basile $
