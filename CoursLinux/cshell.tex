% fichier cshell.tex
% sous  github.../bstarynk/misc-basile/CoursLinux
% licence CC-BY-SA
% Copyright 2025 Basile STARYNKEVITCH, 92340 Bourg-la-Reine
%
% need debian package texlive-latex-extra & texlive-bibtex-extra &
% texlive-full
% cf https://www.unilim.fr/pages_perso/stephane.vinatier/LaTeX/guide-lualatex.pdf
% cf https://fr.overleaf.com/gallery/tagged/beamer
% cf https://tex.stackexchange.com/a/527167
\documentclass[lualatex,11pt,a4paper,svgnames,french]{beamer}
%https://deic.uab.cat/~iblanes/beamer_gallery/
\usetheme{AnnArbor}
%\usepackage[T1]{fontenc}
% inputenc is not needed with lualatex
% \usepackage[utf8x]{inputenc}
\usepackage{alltt}
% https://tex.stackexchange.com/a/342804/42406
% IMPORTANT! see also: https://tex.stackexchange.com/a/500963/42406
%\usepackage{textcomp}
%\usepackage{moreverb}
%\usepackage{fancyvrb}
%\usepackage{fancyhdr}
%\usepackage{fancybox}
% https://tex.stackexchange.com/a/226497/42406
%\usepackage[title]{appendix}
% libertine, see https://tex.stackexchange.com/a/9868/42406
%\usepackage{libertine}
%\usepackage{epsfig}
%\usepackage{graphicx}
%\usepackage{float}
%\usepackage{xcolor}
%\usepackage{moreverb}
%\usepackage{multirow}
%\usepackage{boxedminipage}
%\usepackage[square]{natbib}
% https://tex.stackexchange.com/a/16992/42406
%\usepackage{mathabx}
%\usepackage{charter}
%\usepackage{inconsolata}
%\usepackage{hevea}
%\usepackage{listings}
\usepackage{relsize}
%\usepackage{verbatimbox}
%\usepackage{verbatim}
%\usepackage{filecontents}
%\usepackage{catchfile}
%\usepackage{lastpage}
%\usepackage{stmaryrd}
%\usepackage{ucs}
%\usepackage{stix}
%\usepackage{newunicodechar}
% bigfoot enables \verb in footnotes
%\usepackage{bigfoot}
%\usepackage{makeidx}
%\usepackage{times}
% https://tex.stackexchange.com/a/413066/42406
%\usepackage{apptools}
%\usepackage[a4paper, margin=2cm]{geometry}
\usepackage{hyperref}

\newcommand{\clbemail}[1]{{\href{mailto:#1}{\texttt{\textbf{\textcolor{Navy}{#1}}}}}}
\newcommand{\clbhref}[2]{{\href{https:#1}{{\textcolor{Navy}{#2}}}}}
\newcommand{\clbman}[2]{{\href{https://man7.org/linux/man-pages/#1.html}{{\textcolor{Navy}{\texttt{#2}}}}}}
\newcommand{\clburl}[1]{{\href{https://#1}{\texttt{\relsize{-1}{\textbf{#1}}}}}}
\newcommand{\clbrougras}[1]{{\textcolor{Red}{\textbf{#1}}}}
\newcommand{\clbshell}[1]{{\textcolor{DarkGreen}{\texttt{#1}}}}
\title{Cours sur le shell de Linux} 
\author{Basile \textsc{Starynkevitch}}
\date{automne 2025}

% see also http://www.sascha-frank.com/Arrow/latex-arrows.html
% and http://tug.ctan.org/info/symbols/comprehensive/symbols-a4.pdf
% and https://ctan.math.illinois.edu/macros/latex/contrib/newunicodechar/newunicodechar.pdf
%%%% keep in order
%U+21A6 RIGHTWARDS ARROW FROM BAR
%\newunicodechar{↦}{$\mapsto$}
%U+21B3 DOWNWARDS ARROW WITH TIP RIGHTWARDS
%\newunicodechar{↳}{\rotatebox[origin=c]{180}{$\Lsh$}}
%U+2208 ELEMENT OF
%\newunicodechar{∈}{$\in$}
% U+00AB LEFT-POINTING DOUBLE ANGLE QUOTATION MARK
%\newunicodechar{«}{\guillemotleft}
% U+00BB RIGHT-POINTING DOUBLE ANGLE QUOTATION MARK
%\newunicodechar{»}{\guillemotright}
% U+00B1 PLUS-MINUS SIGN
%\newunicodechar{±}{$\pm$}
% U+00B5 MICRO SIGN
%\newunicodechar{µ}{$\mu$}

\include{gener-macro}

\begin{document}

\begin{frame}
\titlepage
\end{frame}

\begin{frame}{Un cours sur le shell Linux}

  par

  \begin{center}Basile STARYNKEVITCH \\
  8, rue de la Faïencerie \\
  92340 Bourg-la-Reine \\
  courriel: \clbemail{basile@starynkevitch.net}
  \end{center}
  \bigskip
  
  généré: \textit{\clbdate} 

  \bigskip
  
  git: \texttt{\clbgitid}

  \begin{center}
    Ce cours contient des références
    \href{https://fr.wikipedia.org/wiki/Raisonnement_circulaire}{\textcolor{Navy}{circulaires}}
    et des
    \href{https://fr.wikipedia.org/wiki/Hyperlien}{\textcolor{Navy}{hyperliens}}
  \end{center}
\end{frame}

\begin{frame}{Plan}
  \tableofcontents
\end{frame}

%%%%%%%%%%%%%%%%%%%%%%%%%%%%%%%%%%%%%%%%%%%%%%%%%%%%%%%%%%%%
%%%%%%%%%%%%%%%%%%%%%%%%%%%%%%%%%%%%%%%%%%%%%%%%%%%%%%%%%%%%
\section{role-shell}
\label{sec:role-shell}
%%%%
\begin{frame}\frametitle{§ \ref{sec:role-shell}}
{\Large \clbrougras{Rôles du shell}}
\end{frame}
%%%%
\begin{frame}\frametitle{Les rôles du shell}

  \begin{itemize}
  \item Le \clbhref{//fr.wikipedia.org/wiki/Shell_Unix}{shell} (ou interface de commande) permet d'utiliser un ordinateur Linux ``sans souris''.
  \item c'est notamment l'interface principale des supercalculateurs (cf \clburl{top500.org}
  \item Le shell sert aussi à lancer des commandes distantes ou différées dans le temps (dans une heure ou tous les vendredis matin)
  \item Le shell est en pratique utilisable par tout programme Linux
  \item Le shell est \clbrougras{un programme Linux comme un autre}
  \item Il a plusieurs shells \clburl{zsh.org} \clbhref{www.gnu.org/software/bash/}{GNU bash}
    \item Il existe une norme POSIX pour le shell
  \end{itemize}
  \end{frame}

%%%
\begin{frame}\frametitle{...}
  {\relsize{+10}{INCOMPLET}}
\end{frame}


\end{document}
