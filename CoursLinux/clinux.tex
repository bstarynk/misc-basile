% fichier clinux.tex
% sous  github.../bstarynk/misc-basile/CoursLinux
% licence CC-BY-SA
% Copyright 2025 Basile STARYNKEVITCH, 92340 Bourg-la-Reine
%
% need debian package texlive-latex-extra & texlive-bibtex-extra &
% texlive-full
% cf https://www.unilim.fr/pages_perso/stephane.vinatier/LaTeX/guide-lualatex.pdf
% cf https://fr.overleaf.com/gallery/tagged/beamer
% cf https://tex.stackexchange.com/a/527167
\documentclass[lualatex,11pt,a4paper,svgnames,french]{beamer}
%https://deic.uab.cat/~iblanes/beamer_gallery/
\usetheme{AnnArbor}
%\usepackage[T1]{fontenc}
% inputenc is not needed with lualatex
% \usepackage[utf8x]{inputenc}
\usepackage{alltt}
% https://tex.stackexchange.com/a/342804/42406
% IMPORTANT! see also: https://tex.stackexchange.com/a/500963/42406
%\usepackage{textcomp}
%\usepackage{moreverb}
%\usepackage{fancyvrb}
%\usepackage{fancyhdr}
%\usepackage{fancybox}
% https://tex.stackexchange.com/a/226497/42406
%\usepackage[title]{appendix}
% libertine, see https://tex.stackexchange.com/a/9868/42406
%\usepackage{libertine}
%\usepackage{epsfig}
%\usepackage{graphicx}
%\usepackage{float}
%\usepackage{xcolor}
%\usepackage{moreverb}
%\usepackage{multirow}
%\usepackage{boxedminipage}
%\usepackage[square]{natbib}
% https://tex.stackexchange.com/a/16992/42406
%\usepackage{mathabx}
%\usepackage{charter}
%\usepackage{inconsolata}
%\usepackage{hevea}
%\usepackage{listings}
\usepackage{relsize}
%\usepackage{verbatimbox}
%\usepackage{verbatim}
%\usepackage{filecontents}
%\usepackage{catchfile}
%\usepackage{lastpage}
%\usepackage{stmaryrd}
%\usepackage{ucs}
%\usepackage{stix}
%\usepackage{newunicodechar}
% bigfoot enables \verb in footnotes
%\usepackage{bigfoot}
%\usepackage{makeidx}
%\usepackage{times}
% https://tex.stackexchange.com/a/413066/42406
%\usepackage{apptools}
%\usepackage[a4paper, margin=2cm]{geometry}
\usepackage{hyperref}

\newcommand{\clbemail}[1]{{\href{mailto:#1}{\texttt{\textbf{\textcolor{Navy}{#1}}}}}}
\newcommand{\clburl}[1]{{\href{https://#1}{\texttt{\relsize{-1}{\textbf{#1}}}}}}
\title{Cours sur Linux} 
\author{Basile \textsc{Starynkevitch}}
\date{automne 2025}

% see also http://www.sascha-frank.com/Arrow/latex-arrows.html
% and http://tug.ctan.org/info/symbols/comprehensive/symbols-a4.pdf
% and https://ctan.math.illinois.edu/macros/latex/contrib/newunicodechar/newunicodechar.pdf
%%%% keep in order
%U+21A6 RIGHTWARDS ARROW FROM BAR
%\newunicodechar{↦}{$\mapsto$}
%U+21B3 DOWNWARDS ARROW WITH TIP RIGHTWARDS
%\newunicodechar{↳}{\rotatebox[origin=c]{180}{$\Lsh$}}
%U+2208 ELEMENT OF
%\newunicodechar{∈}{$\in$}
% U+00AB LEFT-POINTING DOUBLE ANGLE QUOTATION MARK
%\newunicodechar{«}{\guillemotleft}
% U+00BB RIGHT-POINTING DOUBLE ANGLE QUOTATION MARK
%\newunicodechar{»}{\guillemotright}
% U+00B1 PLUS-MINUS SIGN
%\newunicodechar{±}{$\pm$}
% U+00B5 MICRO SIGN
%\newunicodechar{µ}{$\mu$}

\include{gener-macro}

\begin{document}

\begin{frame}
\titlepage
\end{frame}

\begin{frame}{Un cours sur Linux}

  par

  \begin{center}Basile STARYNKEVITCH \\
  8, rue de la Faïencerie \\
  92340 Bourg-la-Reine \\
  courriel: \clbemail{basile@starynkevitch.net}
  \end{center}
  \bigskip
  
  généré: \textit{\clbdate} 

  \bigskip
  
  git: \texttt{\clbgitid}
\end{frame}

\begin{frame}{Plan}
  \tableofcontents
\end{frame}

%%%%%%%%%%%%%%%%%%%%%%%%%%%%%%%%%%%%%%%%%%%%%%%%%%%%%%%%%%%%
%%%%%%%%%%%%%%%%%%%%%%%%%%%%%%%%%%%%%%%%%%%%%%%%%%%%%%%%%%%%
\section{Pourquoi j'apprécie Linux} % why I like Linux
\label{sec:like-linux}
%%%%
\begin{frame}\frametitle{Pourquoi j'apprécie Linux}

  \begin{itemize}
  \item Par habitude
  \item Linux est largement utilisé: des tablettes (ou
    \href{https://www.raspberrypi.org/}{RaspBerryPi}) ou
    téléphones\footnote{Android est une variante de Linux, et les
    FairPhones tournent sous une variante plus libre nommée
    \texttt{/e/OS}}, à l'affichage des gares et des bus, probablement
    dans beaucoup de voitures automobiles\footnote{Pour les fonctions
    non-critiques telles que le GPS ou la radio!}, la plupart des
    serveurs Web, aux superordinateurs\footnote{Tous les
    superordinateurs listés sur \clburl{top500.org} tournent sous
    Linux mais valent des millions d'€ et consomment parfois des MW
    électriques.}
  \item Linux est composé de \textcolor{red}{logiciels libres}\footnote{Mais beaucoup
  d'entreprises vendent des logiciels propriétaires ou des services commerciaux sur
  Linux}, voir aussi \clburl{linuxfoundation.org},
    \clburl{april.org}, \clburl{aful.org}, \clburl{fsfe.org}

  \end{itemize}
  \end{frame}

%%%%
\begin{frame}\frametitle{Code source et code binaire}

Le matériel informatique\footnote{Une carte mère actuelle contient en
2025 des millions d'octets de code binaire, son ``firmware'' ou
``micrologiciel''} ne comprend que (ou ``exécute'') du \textbf{code
  binaire} ou \textbf{code machine}.

\medskip

Le dévelopeur informaticien ne comprend que du \textbf{code source}
qui est tapé au clavier\footnote{Un dévelopeur utilise aussi une
souris d'ordinateur, qui contient, comme son clavier, du code
binaire. Certains dévelopeurs sont aveugles.\\} et lu à
l'écran. Empiriquement \textbf{\textcolor{red}{un dévelopeur}} à temps
plein \textbf{produit} \textbf{\textcolor{red}{quelques dizaines de
    milliers de lignes de code source par an}}.

\medskip

Dans le détail c'est très compliqué, et \textbf{il existe} des codes binaires
ou \textbf{des logiciels qui manipulent, analysent, modifient} voire
améliorent \textbf{du code source ou du code binaire}.


\medskip

Le code source est la forme préférée par le programmeur, et la plus chère.
\end{frame}

%%%%
\begin{frame}\frametitle{Variété des logiciels et des projets}
  Certains logiciels sont critiques:
  \begin{itemize}
  \item logiciel (et matériel) de pilotage d'un métro automatique
  \item logiciel d'un respirateur Covid (projet
    \href{https://github.com/Recovid/}{\textcolor{Navy}{Recovid}}).
  \end{itemize}
  D'autres le sont moins, mais on d'autres contraintes:
  \begin{itemize}
  \item logiciel de jeu  {\relsize{-1}{(doit être fini pour avant Noël)}}
  \item logiciel de comptabilité {\relsize{-1}{(doit suivre les contraintes et les normes légales)}}
  \item logiciel de bureautique (projet \clburl{libreoffice.org})
  \item logiciel
    \href{https://fr.wikipedia.org/wiki/Compilateur}{\textcolor{Navy}{\textit{compilateur}}}
    {\relsize{-1}{(traduisant le code source en binaire)}} dont \clburl{gcc.gnu.org}
    et \clburl{ocaml.org}  {\relsize{-1}{(doit suivre les usages et les standards)}}
  \item projet scolaire ou universitaire
  \end{itemize}

  La
  \href{https://fr.wikipedia.org/wiki/Souveraineté_numérique}{\textit{\textcolor{Navy}{souveraineté
        numérique}}} est un argument pour le développement, le
  financement, et l'amélioration de logiciels libres...
  
\end{frame}


%%%%
\begin{frame}\frametitle{Obligations légales et de bon sens}

  Se comporter en \href{https://fr.wikipedia.org/wiki/Bon_père_de_famille}{\textit{\textcolor{Navy}{bon père de famille}}} et en
  \href{https://fr.wikipedia.org/wiki/Proffessionnel}{\textit{\textcolor{Navy}{professionnel}}}

      \begin{itemize}
      \item en France: articles 323 et suivants du code pénal.
      \item ne pas coder volontairement des \href{https://fr.wikipedia.org/wiki/Logiciel_malveillant}{logiciels malveillants}
        \item
          \href{https://fr.wikipedia.org/wiki/Loi_de_Murphy_(homonymie)}{loi
            de Murphy}: les catastrophes arrivent au mauvais moment.
        \item définir les données et les codes qui vous sont chers.
          \item sauvegarder ceux ci (notamment le code source qui vous
            co-dévelopez) periodiquement (3 fois par semaine),
            peut-être à distance.
            \item nettoyez préventivement plusieurs fois par ans vos
              ordinateurs (ils prennent la poussière et brûlent).
      \end{itemize}

      Savoir que \textbf{tout logiciel est bogué}, donc \textbf{il
        faut documenter votre travail}.
\end{frame}



%%%%
\begin{frame}\frametitle{Distributions Linux}

  Elles contiennent un grand nombre de \textcolor{red}{\textbf{paquets\footnote{plusieurs milliers de
  paquets; il est souhaitable, une fois que vous maitrisez Linux dans
  un cadre professionnel, d'y contribuer, en installant un
  serveur de paquet d'une distribution sur le serveur qui vous sera
  affecté.\\} logiciels}}. Sur une
  clef USB (ou des CDROMs) on trouve de quoi installer Linux sur un
  ordinateur connecté à Internet. L'installation prend quelques heures
  et va télécharger d'autres paquets depuis un serveur distant.

  \medskip
  
  Beaucoup de
  \href{https://fr.wikipedia.org/wiki/Distribution_Linux}{\textcolor{Navy}{\textit{distributions}}}
  GNU/Linux sont facilement téléchargeables et redistribuables:
    \begin{itemize}
    \item \textbf{Debian} sur \clburl{debian.org}
    \item \textbf{Ubuntu} sur \clburl{ubuntu.com}, dérivée de Debian,
      et sa variante
      \href{https://fr.wikipedia.org/wiki/GendBuntu}{\textcolor{Navy}{GEndBuntu}}
    \item \textbf{Mageia} sur \clburl{mageia.org}
    \item \textbf{RedHat} sur \clburl{redhat.com}
    \end{itemize}
\end{frame}
%%%%
\begin{frame}\frametitle{Variété des distributions}

Il existe des distributions non linuxiennes mais plus ou moins libres:
\clburl{freebsd.org}, \clburl{openbsd.org}, \clburl{hurd.gnu.org},
\clburl{tunes.org}, \href{https://sources.vsta.org:7100/}{VSTa},
\clburl{templeos.org}, \clburl{illumos.org} etc...

\medskip

Dans le monde des distributions Linux on peut distinguer:

\begin{relsize}{-0.6}
\begin{itemize}
\item les \textbf{distributions \textcolor{red}{binaires}}
  (\clburl{ubuntu.com} comme \clburl{linuxmint.com} et beaucoup
  d'autres, souvent dérivées de \clburl{debian.org}, et
  \clburl{redhat.com}) qui existent sous plusieurs versions de
  stabilité et nouveauté variées. Elles sont recommandées pour les
  débutants (ou ceux sont pressés d'avoir un ordinateur utilisable
  sous Linux). Plusieurs distributions facilitent\footnote{En
  pratique, à chaque redémarrage, on choisit au clavier quel système
  va démarrer.\medskip} la cohabitation sur un ordinateur de Linux
  avec Windows.

\item les \textbf{distributions \textcolor{red}{sources}}
  (\clburl{gentoo.org} et \clburl{archlinux.org} parmi d'autres) qui
  compilent les paquets logiciels à leur installation. Elles
  s'adressent à des experts, et leur installation prendra du temps
  mais peut être optimale.
\end{itemize}
\end{relsize}
\end{frame}



%%%%
\begin{frame}\frametitle{Libertés du logiciel libre}

Elles sont définies dans différentes licences logicielles,
principalement La
\href{https://fr.wikipedia.org/wiki/Licence_publique_générale_GNU}{licence
  publique GNU} GPL
\href{https://www.gnu.org/licenses/gpl-3.0.html}{3.0} (en
\clburl{www.gnu.org/licenses/}), la licence LGPL, les licences
\clburl{cecill.info} ou \clburl{eupl.eu}, et d'autres.

La GPL est ma licence logicielle préférée, et permet:
\begin{itemize}
\item {\relsize{-1}{(liberté 0)}} la \textbf{\textcolor{red}{liberté
    d'exécuter}} le logiciel pour n'importe quel usage;
\item {\relsize{-1}{(liberté 1)}} la  \textbf{\textcolor{red}{liberté d'étudier le
  fonctionnement}} d'un programme et de l'adapter à ses besoins, ce qui
  passe par l'accès aux codes sources;
\item {\relsize{-1}{(liberté 2)}} la  \textbf{\textcolor{red}{liberté de redistribuer}} des
  copies du logiciel;
\item {\relsize{-1}{(liberté 3)}} \textbf{\textcolor{red}{la liberté de faire
  bénéficier\footnote{Il y a alors l'obligation morale voire légale de
  redistribuer le code source du logiciel amélioré sous la même
  licence, c'est le
  \href{https://fr.wikipedia.org/wiki/Copyleft}{\textcolor{red}{copyleft}}!} la
  communauté}} des versions modifiées.
\end{itemize}

\smallskip

\begin{relsize}{-1}
  Un logiciel propriétaire a sa licence et ses obligations légales. Sa
  redistribution est souvent interdite, son exécution, sa
  modification, son analyse est limitée par contrat.
\end{relsize}
\end{frame}

%%%%
\begin{frame}\frametitle{Économie du logiciel libre}
Ils ne sont \textbf{pas gratuits}

\medskip

C'est un
\href{https://fr.wikipedia.org/wiki/Bien_commun}{\textcolor{Navy}{bien
    commun}} (cf livres et publications de
\href{https://fr.wikipedia.org/wiki/Jean_Tirole}{\textcolor{Navy}{Jean
    Tirole}}).

\medskip

La difficulté est d'organiser et de faire vivre une \href{https://fr.wikipedia.org/wiki/Communauté_du_logiciel_libre}{\textcolor{Navy}{communauté}}. 

\medskip

Il peut exister des financements institutionnels
\href{https://fr.wikipedia.org/wiki/Horizon_Europe}{\textcolor{Navy}{Horizon
    Europe}} ou
\href{https://fr.wikipedia.org/wiki/Agence_nationale_de_la_recherchee}{\textcolor{Navy}{ANR}}
ou \clburl{itea4.org} et il faut les favoriser

\medskip

Voir le moteur d'inférences libre
\href{https://github.com/RefPerSys/RefPerSys/}{\textcolor{Navy}{RefPerSys}}
(``reflexive persistent system'') et les livres (et logiciels) de
\href{https://fr.wikipedia.org/wiki/Jacques_Pitrat}{\textcolor{Navy}{Jacques
    Pitrat}}

\end{frame}


%%%%%%%%%%%%%%%%%%%%%%%%%%%%%%%%%%%%%%%%%%%%%%%%%%%%%%%%%%%%%%%%
%%%%%%%%%%%%%%%%%%%%%%%%%%%%%%%%%%%%%%%%%%%%%%%%%%%%%%%%%%%%%%%%
\section{Démarrage d'un système Linux} % How Linux starts
\label{sec:start-linux}
%%%%
\begin{frame}\frametitle{Démarrage d'un système Linux déjà installé}

  Comment démarre à froid un ordinateur Linux\footnote{Le démarrage
  rarissime d'un supercalculateur est très complexe et je ne pourrais
  pas l'expliquer en détail.} (portable ou fixe), où une distribution
  Linux a déjà été installée il y a plusieurs mois.

  \begin{enumerate}
  \item le micrologiciel
    %(de nos jours,
    %\href{https://fr.wikipedia.org/wiki/UEFI}{\textclolor{Navy}{UEFI}})
    de la carte mère démarre et initialise le matériel. Une
    \href{https://fr.wikipedia.org/wiki/Mémoire_flash}{\textcolor{Navy}{mémoire
        flash}} contient le paramétrage pour ça.
    \item un chargeur (``boot-loader'', un code binaire de quelques kilo-octets) est
      chargé dans la mémoire vive; c'est le logiciel libre \clburl{www.gnu.org/software/grub} codé en assembleur et C.
  \end{enumerate}
\end{frame}
\end{document}
