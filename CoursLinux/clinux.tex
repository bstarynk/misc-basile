% fichier clinux.tex
% sous  github.../bstarynk/misc-basile/CoursLinux
% licence CC-BY-SA
% Copyright 2025 Basile STARYNKEVITCH, 92340 Bourg-la-Reine
%
% need debian package texlive-latex-extra & texlive-bibtex-extra &
% texlive-full
% cf https://www.unilim.fr/pages_perso/stephane.vinatier/LaTeX/guide-lualatex.pdf
% cf https://fr.overleaf.com/gallery/tagged/beamer
% cf https://tex.stackexchange.com/a/527167
\documentclass[lualatex,11pt,a4paper,svgnames,french]{beamer}
%https://deic.uab.cat/~iblanes/beamer_gallery/
\usetheme{AnnArbor}
%\usepackage[T1]{fontenc}
% inputenc is not needed with lualatex
% \usepackage[utf8x]{inputenc}
\usepackage{alltt}
% https://tex.stackexchange.com/a/342804/42406
% IMPORTANT! see also: https://tex.stackexchange.com/a/500963/42406
%\usepackage{textcomp}
%\usepackage{moreverb}
%\usepackage{fancyvrb}
%\usepackage{fancyhdr}
%\usepackage{fancybox}
% https://tex.stackexchange.com/a/226497/42406
%\usepackage[title]{appendix}
% libertine, see https://tex.stackexchange.com/a/9868/42406
%\usepackage{libertine}
%\usepackage{epsfig}
%\usepackage{graphicx}
%\usepackage{float}
%\usepackage{xcolor}
%\usepackage{moreverb}
%\usepackage{multirow}
%\usepackage{boxedminipage}
%\usepackage[square]{natbib}
% https://tex.stackexchange.com/a/16992/42406
%\usepackage{mathabx}
%\usepackage{charter}
%\usepackage{inconsolata}
%\usepackage{hevea}
%\usepackage{listings}
%\usepackage{relsize}
%\usepackage{verbatimbox}
%\usepackage{verbatim}
%\usepackage{filecontents}
%\usepackage{catchfile}
%\usepackage{lastpage}
%\usepackage{stmaryrd}
%\usepackage{ucs}
%\usepackage{stix}
%\usepackage{newunicodechar}
% bigfoot enables \verb in footnotes
%\usepackage{bigfoot}
%\usepackage{makeidx}
%\usepackage{times}
% https://tex.stackexchange.com/a/413066/42406
%\usepackage{apptools}
%\usepackage[a4paper, margin=2cm]{geometry}
\usepackage{hyperref}

\newcommand{\clbemail}[1]{{\href{mailto:#1}{\texttt{\textbf{\textcolor{Navy}{#1}}}}}}
\newcommand{\clburl}[1]{{\href{#1}{\texttt{\relsize{-1}{\textbf{#1}}}}}}
\title{Cours sur Linux} 
\author{Basile \textsc{Starynkevitch}}
\date{automne 2025}

% see also http://www.sascha-frank.com/Arrow/latex-arrows.html
% and http://tug.ctan.org/info/symbols/comprehensive/symbols-a4.pdf
% and https://ctan.math.illinois.edu/macros/latex/contrib/newunicodechar/newunicodechar.pdf
%%%% keep in order
%U+21A6 RIGHTWARDS ARROW FROM BAR
%\newunicodechar{↦}{$\mapsto$}
%U+21B3 DOWNWARDS ARROW WITH TIP RIGHTWARDS
%\newunicodechar{↳}{\rotatebox[origin=c]{180}{$\Lsh$}}
%U+2208 ELEMENT OF
%\newunicodechar{∈}{$\in$}
% U+00AB LEFT-POINTING DOUBLE ANGLE QUOTATION MARK
%\newunicodechar{«}{\guillemotleft}
% U+00BB RIGHT-POINTING DOUBLE ANGLE QUOTATION MARK
%\newunicodechar{»}{\guillemotright}
% U+00B1 PLUS-MINUS SIGN
%\newunicodechar{±}{$\pm$}
% U+00B5 MICRO SIGN
%\newunicodechar{µ}{$\mu$}

\include{gener-macro}

\begin{document}

\begin{frame}
\titlepage
\end{frame}

\begin{frame}{Un cours sur Linux}

  par

  \begin{center}Basile STARYNKEVITCH \\
  8, rue de la Faïencerie \\
  92340 Bourg-la-Reine \\
  courriel: \clbemail{basile@starynkevitch.net}
  \end{center}
  \bigskip
  
  généré: \textit{\clbdate} 

  \bigskip
  
  git: \texttt{\clbgitid}
\end{frame}

\begin{frame}{Plan}
  \tableofcontents
\end{frame}

\section{Pourquoi j'apprécie Linux} 
\begin{frame}\frametitle{Pourquoi j'apprécie Linux}

  \begin{itemize}
  \item Par habitude
  \item Linux est largement utilisé: des tablettes (ou RaspBerryPi) ou
    téléphones, à l'affichage des gares et des bus, la plupart des
    serveurs Web, aux superordinateurs\footnote{Tous les
    superordinateurs listés sur \url{top500.org} tournent sous Linux}
  \item Linux est composé de \textcolor{red}{logiciels libres}\footnote{Mais beaucoup
  d'entreprises vendent des logiciels propriétaires ou des services commerciaux sur
  Linux}, voir aussi \url{linuxfoundation.org}, \url{april.org}, \url{aful.org}
  \item les distributions GNU/Linux sont facilement téléchargeables et redistribuables:
    \begin{itemize}
    \item \textbf{Debian} sur \url{debian.org}
    \item \textbf{Ubuntu} sur \url{ubuntu.com}
    \item \textbf{Mageia} sur \url{mageia.org}
    \item \textbf{RedHat} sur \url{redhat.com}
    \item etc.... etc.... 
    \end{itemize}
  \end{itemize}
  
\end{frame}
\end{document}
